%!TEX TS-program = xelatex
%!TEX encoding = UTF-8 Unicode
% Awesome CV LaTeX Template for CV/Resume
%
% This template has been downloaded from:
% https://github.com/posquit0/Awesome-CV
%
% Author:
% Claud D. Park <posquit0.bj@gmail.com>
% http://www.posquit0.com
%
%
% Adapted to be an Rmarkdown template by Mitchell O'Hara-Wild
% 23 November 2018
%
% Template license:
% CC BY-SA 4.0 (https://creativecommons.org/licenses/by-sa/4.0/)
%
%-------------------------------------------------------------------------------
% CONFIGURATIONS
%-------------------------------------------------------------------------------
% A4 paper size by default, use 'letterpaper' for US letter
\documentclass[11pt, a4paper]{awesome-cv}

% Configure page margins with geometry
\geometry{left=1.4cm, top=.8cm, right=1.4cm, bottom=1.8cm, footskip=.5cm}

% Specify the location of the included fonts
\fontdir[fonts/]

% Color for highlights
% Awesome Colors: awesome-emerald, awesome-skyblue, awesome-red, awesome-pink, awesome-orange
%                 awesome-nephritis, awesome-concrete, awesome-darknight

\definecolor{awesome}{HTML}{009ACD}

% Colors for text
% Uncomment if you would like to specify your own color
% \definecolor{darktext}{HTML}{414141}
% \definecolor{text}{HTML}{333333}
% \definecolor{graytext}{HTML}{5D5D5D}
% \definecolor{lighttext}{HTML}{999999}

% Set false if you don't want to highlight section with awesome color
\setbool{acvSectionColorHighlight}{true}

% If you would like to change the social information separator from a pipe (|) to something else
\renewcommand{\acvHeaderSocialSep}{\quad\textbar\quad}

\def\endfirstpage{\newpage}

%-------------------------------------------------------------------------------
%	PERSONAL INFORMATION
%	Comment any of the lines below if they are not required
%-------------------------------------------------------------------------------
% Available options: circle|rectangle,edge/noedge,left/right

\photo{eu\_2021.png}
\name{Thiago Cesar L. Silveira}{}

\position{Pesquisador \textbar{} Biólogo \textbar{} Dr.~Zoologia}
\address{Florianópolis - Brasil\\
Brasileiro - 18/05/1978}

\mobile{+55 48 99113 5385}
\email{\href{mailto:silveira.tcl@gmail.com}{\nolinkurl{silveira.tcl@gmail.com}}}
\homepage{\url{https://www.tsilveira.eco.br/}}
\orcid{0000-0002-3163-479X}
\researchgate{Thiago\_Cesar\_Silveira2}

% \gitlab{gitlab-id}
% \stackoverflow{SO-id}{SO-name}
% \skype{skype-id}
% \reddit{reddit-id}


\usepackage{booktabs}

\providecommand{\tightlist}{%
	\setlength{\itemsep}{0pt}\setlength{\parskip}{0pt}}

%------------------------------------------------------------------------------


\usepackage{float} \usepackage{multicol} \usepackage{colortbl}

\arrayrulecolor{white}

\usepackage{hhline}

\definecolor{light-gray}{gray}{0.95}

% Pandoc CSL macros
\newlength{\cslhangindent}
\setlength{\cslhangindent}{1.5em}
\newlength{\csllabelwidth}
\setlength{\csllabelwidth}{3em}
\newenvironment{CSLReferences}[3] % #1 hanging-ident, #2 entry spacing
 {% don't indent paragraphs
  \setlength{\parindent}{0pt}
  % turn on hanging indent if param 1 is 1
  \ifodd #1 \everypar{\setlength{\hangindent}{\cslhangindent}}\ignorespaces\fi
  % set entry spacing
  \ifnum #2 > 0
  \setlength{\parskip}{#2\baselineskip}
  \fi
 }%
 {}
\usepackage{calc}
\newcommand{\CSLBlock}[1]{#1\hfill\break}
\newcommand{\CSLLeftMargin}[1]{\parbox[t]{\csllabelwidth}{#1}}
\newcommand{\CSLRightInline}[1]{\parbox[t]{\linewidth - \csllabelwidth}{#1}}
\newcommand{\CSLIndent}[1]{\hspace{\cslhangindent}#1}

\begin{document}

% Print the header with above personal informations
% Give optional argument to change alignment(C: center, L: left, R: right)
\makecvheader

% Print the footer with 3 arguments(<left>, <center>, <right>)
% Leave any of these blank if they are not needed
% 2019-02-14 Chris Umphlett - add flexibility to the document name in footer, rather than have it be static Curriculum Vitae
\makecvfooter
  {November 2021}
    {Thiago Cesar L. Silveira~~~·~~~Curriculum Vitae}
  {\thepage}


%-------------------------------------------------------------------------------
%	CV/RESUME CONTENT
%	Each section is imported separately, open each file in turn to modify content
%------------------------------------------------------------------------------



\hypertarget{minha-caminhada}{%
\section{Minha Caminhada}\label{minha-caminhada}}

\begin{center}\includegraphics{Abad_template_files/figure-latex/edu_plot-1} \end{center}

\hypertarget{sobre-mim}{%
\section{Sobre mim}\label{sobre-mim}}

Possuo graduação em Ciências Biológicas, Mestrado em Biologia Animal
(UFRGS) e Doutorado em Zoologia (PUCRS - Doutorado Sanduíche/Gales-UK).
Atualmente sou pós-doc no Programa de Pós-Graduação em Ecologia da UFSC
junto ao \href{https://labarufsc.weebly.com/}{\emph{Laboratório de
Abientes Recifais - LABAR}}. Tenho experiência na área de ecologia
aquática com ênfase em invertebrados, peixes, invasões biológicas,
modelagem espaciaal e de nicho. Tenho experiência na produção de
trabalhos técnicos envolvendo inventários de fauna aquática marinha e
análise de bancos de dados. Minha experiência de ensino é em nível de
pós-graduação, variando entre estatística aplicada à ecologia, bancos de
dados e práticas ecológicas.

\hypertarget{formauxe7uxe3o}{%
\section{Formação}\label{formauxe7uxe3o}}

\begin{cventries}
    \cventry{PUCRS}{Ciências Biológicas}{Porto Alegre-RS}{1999-2004}{}\vspace{-4.0mm}
    \cventry{UFRGS}{Mestrado em Biologia Animal}{Porto Alegre-RS}{2005-2007}{}\vspace{-4.0mm}
    \cventry{PUCRS}{Doutorado em Zoologia (Período Sanduíche - Bangor - UK)}{Porto Alegre-RS}{2011-2015}{}\vspace{-4.0mm}
\end{cventries}

\hypertarget{formauxe7uxe3o-complementar}{%
\section{Formação Complementar}\label{formauxe7uxe3o-complementar}}

\begin{cventries}
    \cventry{UFRGS}{Introdução à Modelagem Hierárquica}{Porto Alegre-RS (45h)}{2019-2019}{}\vspace{-4.0mm}
    \cventry{PADI}{Mergulho Avançado - PADI}{Florianópolis-SC (16h)}{2017-2017}{}\vspace{-4.0mm}
    \cventry{BANGOR University}{Research Survey on the Research Vessel Prince Madog}{Bangor-UK (100h)}{2014-2014}{}\vspace{-4.0mm}
    \cventry{Imperial College}{Species distributions models: concepts, methods, applications, and challenges}{Ascot-UK (75h)}{2014-2014}{}\vspace{-4.0mm}
    \cventry{GEOSABER}{Geoprocessamento com o SIG livre QUANTUM GIS}{São Paulo-SP (24h)}{2013-2013}{}\vspace{-4.0mm}
\end{cventries}

\hypertarget{publicauxe7uxf5es-selecionadas}{%
\section{Publicações
Selecionadas}\label{publicauxe7uxf5es-selecionadas}}

\begin{cventries}
    \cventry{Fontoura NF, Alvez TP, Silveira TCL}{A distribuição de peixes e invertebrados no lago Guaíba como subsídio para o licenciamento ambiental.}{1. ed. Porto Alegre: Edipucrs, 164p.}{2021}{}\vspace{-4.0mm}
    \cventry{Tagliari, MM, Vieilledent G, Alaves J ; SILVEIRA TCL, PERONI, N}{Relict populations of Araucaria angustifolia will be isolated, poorly protected, and unconnected under climate and land-use change in Brazil}{Biodiversity and Conservation, 30, 3665}{2021}{}\vspace{-4.0mm}
    \cventry{Crivellaro M, Silveira TCL, Custodio FY, Bataglim L, Deuchon M, Carvalhal A. Ramos BS}{Fighting on the edge: reproductive effort and population structure of the invasive coral Tubastraea coccinea in its southern Atlantic limit of distribution following control activities.}{Bilogical Invasions 23, 811}{2020}{}\vspace{-4.0mm}
    \cventry{das Neves Lopes M, Decarli CJ, Pinheiro-Silva L, Silveira TCL, Leite NK, Petrucio MM}{Urbanization increases carbon concentration and pCO2 in subtropical streams.}{Environ Sci Pollut Res, 1}{2020}{}\vspace{-4.0mm}
    \cventry{Gouvêa LP, Assis J, Gurgel CFD, Serrão EA, Silveira TCL, Santos R, et al.}{Golden carbon of Sargassum forests revealed as an opportunity for climate change mitigation.}{Sci Total Environ, 138745}{2020}{}\vspace{-4.0mm}
    \cventry{Fontoura NF, Schulz UH, Alves TP, Silveira TCL, Pereira JJ, Antonetti DA.}{How Far Upstream: A Review of Estuary-Fresh Water Fish Movements in a Large Neotropical Basin.}{Front Mar Sci,6,39.}{2019}{}\vspace{-4.0mm}
    \cventry{Agrelo M, Daura-Jorge FG, Bezamat C, Silveira TCL, Volkmer de Castilho P, Rodrigues Pires JS, et al.}{Spatial behavioural response of coastal bottlenose dolphins to habitat disturbance in southern Brazil}{Aquat Conserv Mar Freshw Ecosyst 29(11), 1949 }{2019}{}\vspace{-4.0mm}
    \cventry{Batista MB, Anderson AB, Sanches PF, Polito PS, Silveira TCL, Velez-Rubio GM, et al.}{Kelps’ long-distance dispersal: Role of Ecological/Oceanographic processes and implications to marine forest conservation}{Diversity, 10(1),11}{2018}{}\vspace{-4.0mm}
    \cventry{Teixeira KO, Silveira TCL, Harter-marques B.}{Different Responses in Geographic Range Shifts and Increase of Niche Overlap in Future.}{Sociobiology 65(4),630}{2018}{}\vspace{-4.0mm}
    \cventry{Dechoum M de S, Suhs RB, Giehl ELH, Silveira TCL, Ziller SR.}{Citizen engagement in the management of non-native invasive trees: does it make a difference?}{Biol Invasions 21(1), 175}{2018}{}\vspace{-4.0mm}
    \cventry{Silveira TCL, Gama AM da S, Alves TP, Fontoura NF.}{Modeling habitat suitability of the invasive clam Corbicula fluminea in a Neotropical shallow lagoon, southern Brazil}{Braz J Biol, 76,718}{2016}{}\vspace{-4.0mm}
    \cventry{Silveira TCL, Rodrigues GG, Souza GPC de, Würdig NL.}{Effect of Typha domingensis cutting : response of benthic macroinvertebrates and macrophyte regeneration.}{Biota Neotrop 12,122}{2012}{}\vspace{-4.0mm}
    \cventry{Silveira TCL, Rodrigues GG, Souza GPC De, Würdig NL.}{Effects of cutting disturbance in Schoenoplectus californicus (C.A. Mey.) Soják on the benthic macroinvertebrates}{Acta Sci Biol Sci, 33,31}{2011}{}\vspace{-4.0mm}
\end{cventries}

\hypertarget{principais-experiuxeancias-docentes}{%
\section{Principais Experiências
Docentes}\label{principais-experiuxeancias-docentes}}

\begin{cventries}
    \cventry{UFSC}{Ecology lectures (Ministrada em Inglês)}{Florianópolis-SC}{2019-2019}{}\vspace{-4.0mm}
    \cventry{UFSC}{Distribuição de Espécies e Modelagem de Nicho (Português)}{Florianópolis-SC}{2016-2018}{}\vspace{-4.0mm}
    \cventry{UFSC}{Estatística Aplicada a Análise Ambiental (Português)}{Florianópolis-SC}{2016-2019}{}\vspace{-4.0mm}
    \cventry{UFSC}{Introducão à Análise Multivariada (Português)}{Florianópolis-SC}{2016-2019}{}\vspace{-4.0mm}
    \cventry{UFSC}{Ecologia de Campo (Português)}{Florianópolis-SC}{2016-2019}{}\vspace{-4.0mm}
    \cventry{UFSC}{Organização de Bancos de Dados para Análises Ecológicas (Português/Curso de Curta Duração)}{Florianópolis-SC}{2017-2017}{}\vspace{-4.0mm}
\end{cventries}

\hypertarget{principais-trabalhos-tuxe9cnicos}{%
\section{Principais Trabalhos
Técnicos}\label{principais-trabalhos-tuxe9cnicos}}

\begin{cventries}
    \cventry{Golder Associates}{Caracterização Ambiental do Trecho 17 do PMR - Ictiofauna, Ictioplâncton e Invertebrados Bentônicos - Impactos dos rejeitos da barragem de Fundão no Ecossitema Marinho}{Espírito Santo}{2020}{}\vspace{-4.0mm}
    \cventry{PUCRS-FEPAM}{A Distribuição de Peixes e Invertebrados no Lago Guaíba como Subsídio para o Licenciamento Ambiental}{Porto Alegre-RS}{2019-2020}{}\vspace{-4.0mm}
    \cventry{MAArE - Petrobrás}{Integração dos Resultados Biológicos. in: Projeto de Monitoramento Ambiental da Reserva Biológica Marinha do Arvoredo e Entorno. Relatório Técnico Final.}{Florianópolis-SC}{2017-2018}{}\vspace{-4.0mm}
    \cventry{NSF-Bioensaios-Petrobrás}{Ictiofauna e Produtividade Pesqueira da Região de Madre de Deus BA - Refinaria Landulpho Alves (RLAM).}{Madre de Deus-BA}{2015-2018}{}\vspace{-4.0mm}
\end{cventries}

\hypertarget{projetos}{%
\section{Projetos}\label{projetos}}

\begin{cventries}
    \cventry{UFSC}{Subsídios para o plano controle do coral invasor coral-sol (Tubastraea coccinea) na Reserva Biológica Marinha do Arvoredo: aspectos reprodutivos e modelagem da distribuição potencial}{SC-Brasil}{2016-Atual}{}\vspace{-4.0mm}
    \cventry{UFSC}{Modelos de Nicho ecológico e Distribuição Potencial de espécies e os efeitos das mudanças climáticas}{SC-Brasil}{2016-Atual}{}\vspace{-4.0mm}
    \cventry{UFSC}{Conhecimento da fauna bentônica marinha do sul do Brasil: quanto sabemos e como se distribui}{SC-Brasil}{2017-Atual}{}\vspace{-4.0mm}
    \cventry{UFSC}{Projeto MAArE - Monitoramento Ambiental da Reserva Biológica Marinha do Arvoredo e Entorno}{SC-Brasil}{2017-2018}{}\vspace{-4.0mm}
    \cventry{PUCRS}{Subsídios à gestão pesqueira do Lago Guaíba, RS: período reprodutivo, tamanho de primeira maturação, seletividade de captura e padrão sazonal de ocupação de hábitat de peixes importância econômica e ictiofauna acompanhante}{SC-Brasil}{2011-2015}{}\vspace{-4.0mm}
\end{cventries}

\hypertarget{habilidades}{%
\section{Habilidades}\label{habilidades}}

\begin{cventries}
    \cventry{R -- HTML -- CSS}{Linguagens}{}{}{}\vspace{-4.0mm}
    \cventry{GLM -- Randon Forests -- Abordagens Multivariadas}{Estatísitca}{}{}{}\vspace{-4.0mm}
    \cventry{QGIS -- RStudio -- Visual Studio Code -- Inkscape -- Mendeley}{Software}{}{}{}\vspace{-4.0mm}
    \cventry{Javascript -- Python -- OpenDrift (Individual Based Moldelling)}{Estudando}{}{}{}\vspace{-4.0mm}
    \cventry{GIT -- Leaflet -- Markdown}{Outras}{}{}{}\vspace{-4.0mm}
\end{cventries}

\begin{cventries}
    \cventry{R -- HTML -- CSS}{Linguagens}{}{}{}\vspace{-4.0mm}
    \cventry{GLM -- Randon Forests -- Abordagens Multivariadas}{Estatísitca}{}{}{}\vspace{-4.0mm}
    \cventry{QGIS -- RStudio -- Visual Studio Code -- Inkscape -- Mendeley}{Software}{}{}{}\vspace{-4.0mm}
    \cventry{Javascript -- Python -- OpenDrift (Individual Based Moldelling)}{Estudando}{}{}{}\vspace{-4.0mm}
    \cventry{GIT -- Leaflet -- Markdown}{Outras}{}{}{}\vspace{-4.0mm}
\end{cventries}

\hypertarget{informauxe7uxf5es-adicionais}{%
\section{Informações Adicionais}\label{informauxe7uxf5es-adicionais}}

Para acessar o CV na plataforma Lattes use o link no topo deste arquivo
ou acesse (\url{http://lattes.cnpq.br/5960267776845701}). Se desejar
alguma referência profissional, entre em contato com a Dra. Andrea
Santarosa Freire
\textbf{(\href{mailto:freireandreas@gmail.com}{\nolinkurl{freireandreas@gmail.com}})},
coordenadora do Programa de Pós-gradução em Ecologia da UFSC.

\end{document}
