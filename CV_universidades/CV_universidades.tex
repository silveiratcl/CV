%!TEX TS-program = xelatex
%!TEX encoding = UTF-8 Unicode
% Awesome CV LaTeX Template for CV/Resume
%
% This template has been downloaded from:
% https://github.com/posquit0/Awesome-CV
%
% Author:
% Claud D. Park <posquit0.bj@gmail.com>
% http://www.posquit0.com
%
%
% Adapted to be an Rmarkdown template by Mitchell O'Hara-Wild
% 23 November 2018
%
% Template license:
% CC BY-SA 4.0 (https://creativecommons.org/licenses/by-sa/4.0/)
%
%-------------------------------------------------------------------------------
% CONFIGURATIONS
%-------------------------------------------------------------------------------
% A4 paper size by default, use 'letterpaper' for US letter
\documentclass[11pt, a4paper]{awesome-cv}

% Configure page margins with geometry
\geometry{left=1.4cm, top=.8cm, right=1.4cm, bottom=1.8cm, footskip=.5cm}

% Specify the location of the included fonts
\fontdir[fonts/]

% Color for highlights
% Awesome Colors: awesome-emerald, awesome-skyblue, awesome-red, awesome-pink, awesome-orange
%                 awesome-nephritis, awesome-concrete, awesome-darknight

\definecolor{awesome}{HTML}{009ACD}

% Colors for text
% Uncomment if you would like to specify your own color
% \definecolor{darktext}{HTML}{414141}
% \definecolor{text}{HTML}{333333}
% \definecolor{graytext}{HTML}{5D5D5D}
% \definecolor{lighttext}{HTML}{999999}

% Set false if you don't want to highlight section with awesome color
\setbool{acvSectionColorHighlight}{true}

% If you would like to change the social information separator from a pipe (|) to something else
\renewcommand{\acvHeaderSocialSep}{\quad\textbar\quad}

\def\endfirstpage{\newpage}

%-------------------------------------------------------------------------------
%	PERSONAL INFORMATION
%	Comment any of the lines below if they are not required
%-------------------------------------------------------------------------------
% Available options: circle|rectangle,edge/noedge,left/right

\photo{foto\_pht.png}
\name{Thiago C.L. Silveira}{}

\position{Pesquisador \textbar{} Biólogo \textbar{} Dr.~Zoologia}
\address{Florianópolis - Brasil\\
Brasileiro - 18/05/1978}

\mobile{+55 48 99113 5385}
\email{\href{mailto:silveira.tcl@gmail.com}{\nolinkurl{silveira.tcl@gmail.com}}}
\homepage{lattes.cnpq.br/5960267776845701}
\orcid{0000-0002-3163-479X}
\researchgate{Thiago\_Cesar\_Silveira2}

% \gitlab{gitlab-id}
% \stackoverflow{SO-id}{SO-name}
% \skype{skype-id}
% \reddit{reddit-id}


\usepackage{booktabs}

% Templates for detailed entries
% Arguments: what when with where why
\usepackage{etoolbox}
\def\detaileditem#1#2#3#4#5{%
\cventry{#1}{#3}{#4}{#2}{\ifx#5\empty\else{\begin{cvitems}#5\end{cvitems}}\fi}\ifx#5\empty{\vspace{-4.0mm}}\else\fi}
\def\detailedsection#1{\begin{cventries}#1\end{cventries}}

% Templates for brief entries
% Arguments: what when with
\def\briefitem#1#2#3{\cvhonor{}{#1}{#3}{#2}}
\def\briefsection#1{\begin{cvhonors}#1\end{cvhonors}}

\providecommand{\tightlist}{%
	\setlength{\itemsep}{0pt}\setlength{\parskip}{0pt}}

%------------------------------------------------------------------------------


\usepackage{multicol} \usepackage{colortbl} \usepackage{hhline} \definecolor{light-gray}{gray}{0.95}
\usepackage{booktabs}
\usepackage{longtable}
\usepackage{array}
\usepackage{multirow}
\usepackage{wrapfig}
\usepackage{float}
\usepackage{colortbl}
\usepackage{pdflscape}
\usepackage{tabu}
\usepackage{threeparttable}
\usepackage{threeparttablex}
\usepackage[normalem]{ulem}
\usepackage{makecell}
\usepackage{booktabs}
\usepackage{longtable}
\usepackage{array}
\usepackage{multirow}
\usepackage{wrapfig}
\usepackage{float}
\usepackage{colortbl}
\usepackage{pdflscape}
\usepackage{tabu}
\usepackage{threeparttable}
\usepackage{threeparttablex}
\usepackage[normalem]{ulem}
\usepackage{makecell}

\begin{document}

% Print the header with above personal informations
% Give optional argument to change alignment(C: center, L: left, R: right)
\makecvheader

% Print the footer with 3 arguments(<left>, <center>, <right>)
% Leave any of these blank if they are not needed
% 2019-02-14 Chris Umphlett - add flexibility to the document name in footer, rather than have it be static Curriculum Vitae
\makecvfooter
  {October, 2020}
    {Thiago C.L. Silveira~~~·~~~Curriculum Vitae}
  {\thepage}


%-------------------------------------------------------------------------------
%	CV/RESUME CONTENT
%	Each section is imported separately, open each file in turn to modify content
%------------------------------------------------------------------------------



\section{Sobre mim}\label{sobre-mim}

Possuo graduação em Ciências Biológicas, mestrado em Biologia Animal
(UFRGS) e doutorado em Zoologia (PUCRS - Doutorado Sanduíche em
Walles-UK). Atualmente trabalho como bolsista de pós-doutorando PNPD no
Programa de Pós-Graduação em Ecologia (UFSC). Tenho experiência na área
de ecologia aquática com ênfase em invertebrados, peixes, modelos
espaciais e modelos de nicho usando software R e SIG em meu campo de
pesquisa.\\
Minha experiência de ensino é em nível de pós-graduação, variando de
estatística aplicada à Ecologia e práticas ecológicas em Ecologia de
Campo.

\section{Habilidades e Interesses}\label{habilidades-e-interesses}

Minha formação é em ciências biológicas, com ênfase em Zoologia e
Ecologia. Como pós-doutorando no PPG Ecologia-UFSC fui incentivado a
lecionar, e, ao longo de cinco anos (2016-atual), essa experiência tem
sido muito enriquecedora na minha formação como professor.

\section{Formação}\label{formauxe7uxe3o}

\detailedsection{\detaileditem{Ciências Biológicas}{1999-2004}{PUCRS}{Porto Alegre, RS}{\empty}\detaileditem{Mestrado em Biologia Animal}{2005-2007}{UFRGS}{Porto Alegre, RS}{\empty}\detaileditem{Doutorado em Zoologia}{2011-2015}{PUCRS}{Porto Alegre, RS}{\empty}}

\section{Experiência Docente}\label{experiuxeancia-docente}

\detailedsection{\detaileditem{Ecology lectures (ministrada em Inglês)}{2019-2019}{UFSC}{Florianópolis-SC}{\empty}\detaileditem{Distribuição de Espécies e Modelagem de Nicho (ministrada em Português)}{2016-2018}{UFSC}{Florianópolis-SC}{\empty}\detaileditem{Estatística Aplicada a Análise Ambiental (ministrada em Português)}{2016-2019}{UFSC}{Florianópolis-SC}{\empty}\detaileditem{Introducão a Análise Multivariate (ministrada em Português)}{2016-2019}{UFSC}{Florianópolis-SC}{\empty}\detaileditem{Ecologia de Campo (ministrada em Português)}{2016-2019}{UFSC}{Florianópolis}{\empty}}
\# Artigos

\section{Informações Adicionais}\label{informauxe7uxf5es-adicionais}

To access my complete CV, please go to my Lattes CV on
\href{http://lattes.cnpq.br/5960267776845701}{\textbf{http://lattes.cnpq.br/5960267776845701}}.
If you want a personal reference, please get in touch with Dra. Andrea
Santarosa Freire
\textbf{(\href{mailto:freireandreas@gmail.com}{\nolinkurl{freireandreas@gmail.com}})},
coordinator of the Post Graduation Program in Ecology - UFSC.

\end{document}
